

\documentclass[12pt,]{article}
\usepackage{zed-csp,graphicx,color}%from
\pagenumbering{roman}
\begin{document}

\begin{titlepage}
\begin{figure}[h]
  \centerline{\small MAKERERE 
  \includegraphics[width=0.1\textwidth]  {muk_log} UNIVERSITY}
\end{figure}
\centerline{COLLEGE OF COMPUTING AND INFORMATION SCIENCES\\}
\paragraph*{•}
\centerline{ACCURATE DATA ENTRY SYSTEM\\}
\paragraph*{•}
\centerline{ By\\}

\centerline{Group\\}
\paragraph*{•}
\centerline{DEPARTMENT OF COMPUTER SCIENCE\\}
\
\centerline{SCHOOL OF COMPUTING AND INFORMATICS TECHNOLOGY\\}
\centerline{A concept paper submitted to the school of computing}

centerline{for the study leading to the proposal in partial fulfillment of the }
\centerline{requirements for the award of the degree of Bachelor of science in computer science}


\paragraph*{•}
\centerline{Supervisor: Ernest Mwebaze\\}
\centerline{ $March,7^{th},2018$\\}

\centerline{prepared by:\\}
\centerline{\begin{tabular}{|c|c|c|c|}
\hline
\textbf{No.}& \textbf{Student Name} & \textbf{RegNo} & \textbf{Signature} \\ \hline
\textit{1}&\textbf{KAZOOBA B LAWRENCE} & \textit{16/U/5830/PS}& \textit{} \\ \hline
\textit{2}&\textbf{AWAT LYNETTE }& \textit{16/U/3973/PS}& \textit{} \\ \hline
\textit{3}&\textbf{GIRAMIA PATRICIA} & \textit{16/U/4831/EVE} & \textit{} \\ \hline
\textit{4}&\textbf{ORIKIRIZA OSCAR}& \textit{16/U/1055}  & \textit{} \\ \hline
\textit{5} &\textbf{}& \textit{}  & \textit{} \\ \hline
\textit{6}&\textbf{ } & \  & \textit{} \\ 
 \hline
\end{tabular}}

\paragraph*{•}
\paragraph*{•}
  \begin{flushright}
  Data Collection Concept,\\
 
 \tableofcontents

  \end{flushright}
\date{\today}
\end{titlepage}

\newpage



\pagenumbering{arabic}
\section{Introduction}
\section{Background of the study}
; Many significant technological changes have occurred in the IT industry since the beginning of the 21st century., Data entry was by means of hardcopy, big data and it was always fed into the books, ledgers. But there has been an outstanding advance in technology with powerful tools like the OdK collect and package, Google cloud computing and platforms and many others which have really helped with entry and computing abilities. But even with all these technology advancements their still some glitches in data entry such as data inconsistency and data invalidation which we seek to curb in this implementation written in this proposal.

\section{
\textbf{Statement of the problem}}
the problem being solved or implemented is Users entering inaccurate data which sometimes results into data inconsistency, data invalidation and so this results into very wrong or insufficient and inaccurate entry of data.

\section{
\textbf{Objectives}}
-	To curb data inconsistency where same data is just stored n different formats lets say duplication of files.
-	To help in data validation ensuring data cleansing and improve data quality.
-	To help users avoid entering inaccurate data and learn how to use a good data entry system.

\section{
\textbf{General objective}}
To help users to generally learn and curb any difficulties that entails data entry and totally master how to enter accurate data.

\section{
\textbf{Specific objective}}
To help in accuracy of data entry.
-to help in ease and convenience during data entry..
-To help in data consistency and data validation especially when entering data.
-To generally help all sorts of data users, literate, illiterate and semi- illiterate users who all enter data.

\section{
\textbf{Scope}}
This research is going to contain samples of inaccurate data, how the solution is going to be implemented. Examples of data inconsistency and also samples of how data can be validated and stored properly so as to avoid "unclean" or wrong data. Sources of big data and how it was done before the 21st century. Examples of the technological advancements of data entry. The research is going to entail all kids of users and the problems they face as they do data entry, such as the very literate, semi-literate and the illiterate.
  
\section{
\textbf{Significance of the study}}
\textit{"No matter how much a user may double-check an entry, mistakes always slip in"}
This study or research should help all kinds of users who do data entry to be better when entering data and to avoid all the difficulties that come with data entry and should also provide all the examples that deal with data entry and big data  before and after the 21st century. The technological advancements that have come with dealing with data entry , the new ways of data entry that are convenient and their glitches and the solutions that will be implemented.

\end{document}

